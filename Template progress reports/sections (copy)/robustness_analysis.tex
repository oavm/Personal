\documentclass[main.tex]{subfiles}

\begin{document}

An important aspect to analyze when doing controller design, is the robustness of the strategy against variations of the parameters that determine the model. In the case of a directional drilling system, there are two main parameters that can be considered to be uncertain, the active weight on bit $\Pi$ and the bit walk angle $\varpi$. In the case of $\Pi$, this uncertainty is related to different factors. In particular, changes in the applied hook-load, the variation of the interaction of the rock with the drillsting and the bit, and the decrease of sharpness of the bit as the trajectory evolves, are some of the factors that may affect the value of this parameter. It has to be mentioned, that all of this changes are usually considered to influence the fluctuation of $\Pi$ slowly with respect to overall evolution of the borehole. Regarding the bit walk angle $\varpi$, this parameter is  present in the process due to its 3D nature, being one of the main factors that causes undesired behaviors such as borehole spiraling. The effect of this parameter will be analyzed and dealt with in subsequent sections, since the terms related to this parameter add nonlinear effects and coupling to the already complex dynamics of the system. 

Considering that the previous controller design was based on a nominal value of weight on bit $\Pi$, to test if the designed strategy is able to cope with the uncertainty of this parameter, it will be considered as $\Pi = \bar{\Pi} + \delta\Pi$, where $\bar{\Pi}$ represents the nominal value of weight on bit.

\subsection{Error dynamics for robust stability analysis and controller design}

In this section, the error dynamics of the system are derived once again, since the change of parameter $\Pi$ affects the structure of the system. Recalling Equation \eqref{eq:systemstatespace}, the states of the system are given by:

\begin{align}
	\begin{bmatrix}
	x_\Theta' \\
	x_\Phi'
	\end{bmatrix} =&
	\begin{bmatrix}
	A_0 & 0 \\
	0 & A_0
	\end{bmatrix}
	\begin{bmatrix}
	x_\Theta(\xi) \\
	x_\Phi(\xi)
	\end{bmatrix} + 
	\begin{bmatrix}
	A_1 & 0 \\
	0 & A_1
	\end{bmatrix}
	\begin{bmatrix}
	x_\Theta(\xi_1) \\
	x_\Phi(\xi_1)
	\end{bmatrix} +
	\begin{bmatrix}
	A_2 & 0 \\
	0 & A_2
	\end{bmatrix}
	\begin{bmatrix}
	x_\Theta(\xi_2) \\
	x_\Phi(\xi_2)
	\end{bmatrix} \nonumber\\
	+&
	\begin{bmatrix}
	B_{0\Theta} & 0\\
	0 & B_{0\Phi}(\Theta,\check{\Theta},\Theta',\check{\Theta}')
	\end{bmatrix} 
	\begin{bmatrix}
	\Gamma_\Theta^* \\
	\Gamma_\Phi^*
	\end{bmatrix} +
	\begin{bmatrix}
	B_{1\Theta} & 0\\
	0 & B_{1\Phi}(\Theta,\check{\Theta})
	\end{bmatrix} 
	\begin{bmatrix}
	\Gamma_\Theta^{*'} \\
	\Gamma_\Phi^{*'}
	\end{bmatrix} + 
	\begin{bmatrix}
	BW \\
	0
	\end{bmatrix},
	\label{eq:systemstatespacerobust}	
\end{align}

and with output equations given by:

	\begin{align}
		y_\Theta &= C_\Theta x_\Theta + D_\Theta \Gamma_\Theta^* + E W_y, \label{eq:output12robust}\\
		y_\Phi &= C_\Phi x_\Phi + D_\Phi \Gamma_\Phi^*	\frac{\sin \check{\Theta}}{\sin \Theta}	.\label{eq:output22robust}	
	\end{align}

Herein matrices $A_0$, $A_1$, $A_2$, $B_{0i}$, $B_{1i}$, $C_i$ and $D_i$ for $i = \Theta, \Phi$ defined the same as in \eqref{eq:systemstatespace}, evaluated at $\Pi = \bar{\Pi} + \delta \Pi$. It has to be note that matrices $B_{0\Phi}$ and $B_{1\Phi}$ are kept (i.e. not introducing the $\alpha$ term) and have a dependency with respect to $\Theta$, $\check{\Theta}$, $\Theta'$ and $\check{\Theta}'$. From here on the dependency is not made explicit. 

As before, an input filter is introduced as:

\begin{equation}
	\Gamma_i^{*'} = -\bar{b}_0\Gamma_i^* - \bar{b}_1 u_i,
\end{equation}

where $\bar{b}_0$ and $\bar{b}_0$ are defined as in \eqref{eq:Constants} and evaluated at  $\bar{\Pi}$. An important remark is, that this input filter will not be able to get rid of the $\Gamma_i^*$-related terms (not even in the case of the inclination dynamics), due to the difference between nominal matrices $\bar{B}_{0i}$ and $\bar{B}_{1i}$ and their real versions. The system dynamics of the system  can be derived (using the actual versions of the state matrices), applying the input filter and including it as a state of the system, as following:


\begin{align}
\begin{bmatrix}
x_\Theta' \\
\Gamma_\Theta^{*'} \\
x_\Phi' \\
\Gamma_\Phi^{*'} 
\end{bmatrix} =&
\begin{bmatrix}
A_0 & (B_{0\Theta} - B_{1\Theta} \bar{b}_0) & 0 & 0\\
0 & -\bar{b}_0 & 0 & 0 \\
0 & 0 & A_0 & (B_{0\Phi} - B_{1\Phi} \bar{b}_0)\\
0 & 0 & 0 & -\bar{b}_0
\end{bmatrix}
\begin{bmatrix}
x_\Theta(\xi) \\
\Gamma_\Theta^{*}(\xi) \\
x_\Phi(\xi) \\
\Gamma_\Phi^{*} (\xi)
\end{bmatrix} + 
\begin{bmatrix}
A_1 & 0 & 0 & 0\\
0 & 0 & 0 & 0 \\
0 & 0 & A_1 & 0 \\
0 & 0 & 0 & 0 \\
\end{bmatrix}
\begin{bmatrix}
x_\Theta(\xi_1) \\
\Gamma_\Theta^{*}(\xi_1) \\
x_\Phi(\xi_1) \\
\Gamma_\Phi^{*} (\xi_1)
\end{bmatrix} \nonumber\\ 
&+\begin{bmatrix}
A_2 & 0 & 0 & 0 \\
0 & 0 & 0 & 0 \\
0 & 0 & A_2 & 0 \\
0 & 0 & 0 & 0 \\
\end{bmatrix}
\begin{bmatrix}
x_\Theta(\xi_2) \\
\Gamma_\Theta^{*}(\xi_2) \\
x_\Phi(\xi_2) \\
\Gamma_\Phi^{*} (\xi_2)
\end{bmatrix} +
\begin{bmatrix}
-B_{1\Theta}\bar{b}_1 & 0 \\
-\bar{b}_1 & 0 \\
0 & -B_{1\Phi}\bar{b}_1 \\
0 & -\bar{b}_1
\end{bmatrix}
\begin{bmatrix}
u_\Theta \\
u_\Phi
\end{bmatrix} +
\begin{bmatrix}
BW \\
0 \\
0 \\
0
\end{bmatrix}.
\end{align}

 The feedforward input can be now defined as:

	\begin{equation}
	u_{ri} = B^T (x'_{ri}(\xi) - \bar{A}_0 x_{ri}(\xi) - \bar{A}_1 x_{ri}(\xi_1) - \bar{A}_2 x_{ri}(\xi_2)).
	\label{eq:feedforwardrobust}
	\end{equation}

This feedforward input can be utilized to define once again a desired input $\Gamma_{id}^*$ as in Equation \eqref{eq:filter2} and define an error coordinate $\Delta \Gamma_i^*$. Then the error dynamics can be defined as in \eqref{eq:errordynamics2robust}, using the fact that $x_i = e_i + x_{ri}$.


\begin{align}
\begin{bmatrix}
e_\Theta' \\
\Delta \Gamma_\Theta^{*'} \\
e_\Phi' \\
\Delta \Gamma_\Phi^{*'}
\end{bmatrix} =&
\begin{bmatrix}
A_0 & (B_{0\Theta} - B_{1\Theta} \bar{b}_1) & 0 & 0\\
0 & -b_0 & 0 & 0 \\
0 & 0 & A_0 & (B_{0\Phi} - B_{1\Phi} \bar{b}_1) \\
0 & 0 & 0 & -b_0
\end{bmatrix}
\begin{bmatrix}
e_\Theta(\xi) \\
\Delta \Gamma_\Theta^{*} (\xi) \\
e_\Phi(\xi) \\
\Delta \Gamma_\Phi^{*} (\xi) 
\end{bmatrix} + 
\begin{bmatrix}
A_1 & 0 & 0 & 0\\
0 & 0 & 0 & 0 \\
0 & 0 & A_1 & 0 \\
0 & 0 & 0 & 0 
\end{bmatrix}
\begin{bmatrix}
e_\Theta(\xi_1) \\
\Delta \Gamma_\Theta^{*} (\xi_1) \\
e_\Phi(\xi_1) \\
\Delta \Gamma_\Phi^{*} (\xi_1) 
\end{bmatrix} \nonumber\\
&+
\begin{bmatrix}
A_2 & 0 & 0 & 0\\
0 & 0 & 0 & 0 \\
0 & 0 & A_2 & 0 \\
0 & 0 & 0 & 0 
\end{bmatrix}
\begin{bmatrix}
e_\Theta(\xi_2) \\
\Delta \Gamma_\Theta^{*} (\xi_2) \\
e_\Phi(\xi_2) \\
\Delta \Gamma_\Phi^{*} (\xi_2) 
\end{bmatrix} + \begin{bmatrix}
-B_{1\Theta} \bar{b}_1 & 0 \\
-\bar{b}_1 & 0 \\
0 & -B_{1\Phi} \bar{b}_1  \\
0 & -\bar{b}_1 
\end{bmatrix}
\begin{bmatrix}
v_\Theta \\
v_\Phi
\end{bmatrix} + 
\begin{bmatrix}
(-B - B_{1\Theta}\bar{b}_1) \\
0 \\
(-B - B_{1\Phi}\bar{b}_1)  \\
0
\end{bmatrix}
\begin{bmatrix}
u_{r\Theta} \\
u_{r\Phi}
\end{bmatrix}
\label{eq:errordynamics2robust}+
\begin{bmatrix}
F_{p\Theta} \\
0 \\
F_{p\Phi} \\
0
\end{bmatrix},
\end{align}

where the perturbation terms $F_{p\Theta}$ and $F_{p\Phi}$ are defined as:

\begin{align}
	F_{p\Theta} &= (B_{0\Theta} - B_{1\Theta} \bar{b}_0)\Gamma_{\Theta d}^* + dA_0 x_{r\Theta} (\xi) + dA_1 x_{r\Theta} (\xi_1) + dA_2 x_{r\Theta} (\xi_2) + BW \label{eq:PerturbationForces1}\\
	F_{p\Phi} &= (B_{0\Phi} - B_{1\Phi} \bar{b}_0)\Gamma_{\Phi d}^* + dA_0 x_{r\Theta} (\xi) + dA_1 x_{r\Theta} (\xi_1) + dA_2 x_{r\Theta} (\xi_2) \label{eq:PerturbationForces2}\\
\end{align}

where $\Delta A_t = A_t - \bar{A}_t$ for $t = 0,1,2$. Then, the state feedback controller $v_i$ is implemented in the same way as in \eqref{eq:controllerlowpasserror}, \eqref{eq:controllerintegralerror} and \eqref{eq:controllerfeedbackerror} i.e.

		\begin{align}
		z_{1i}' &= \zeta \begin{bmatrix} 
		k_{1i} & 0 & 0
		\end{bmatrix}(\check{x}_i - x_{ri}) \\
		z_{2i}' &= -\gamma z_{2i}  + \gamma (z_{1i} + K_i(\check{x}_i - x_{ri})) \\
		v_i &= z_{2i},
		\end{align}

Accounting for the observer design, the same integral action will be included and defined as in \eqref{eq:observerintegral}, namely,

	\begin{equation}
	q_i' = \zeta[l_{1i},l_{2i}](y_i - \check{y}_i). \label{eq:observerintegral2} \nonumber
	\end{equation} 
	 
Then the observer dynamics can be defined as:

\begin{align}
\begin{bmatrix}
\check{x}_\Theta' \\
q_\Theta' \\
\check{x}_\Phi'\\
q_\Phi'
\end{bmatrix} &=
\begin{bmatrix}
\bar{A}_0 & 0 & 0 & 0\\
0 & 0 & 0 & 0\\
0 & 0 & \bar{A}_0 & 0 \\
0 & 0 & 0 & 0 \\
\end{bmatrix}
\begin{bmatrix}
\check{x}_\Theta(\xi) \\
q_\Theta(\xi) \\
\check{x}_\Phi(\xi) \\
q_\Phi (\xi)
\end{bmatrix} + 
\begin{bmatrix}
\bar{A}_1 & 0 & 0 & 0\\
0 & 0 & 0 & 0\\
0 & 0 & \bar{A}_1 & 0 \\
0 & 0 & 0 & 0 \\
\end{bmatrix}
\begin{bmatrix}
\check{x}_\Theta(\xi_1) \\
q_\Theta(\xi_1) \\
\check{x}_\Phi(\xi_1) \\
q_\Phi(\xi_1) \\
\end{bmatrix} +
\begin{bmatrix}
\bar{A}_2 & 0 & 0 & 0\\
0 & 0 & 0 & 0\\
0 & 0 & \bar{A}_2 & 0 \\
0 & 0 & 0 & 0 \\
\end{bmatrix}
\begin{bmatrix}
\check{x}_\Theta(\xi_2) \\
q_\Theta(\xi_2) \\
\check{x}_\Phi(\xi_2) \\
q_\Phi(\xi_2) 
\end{bmatrix} \nonumber\\
&+
\begin{bmatrix}
L_\Theta(y_{\Theta} - \check{y}_{\Theta})\\
\zeta[l_{1\Theta},l_{2\Theta}](y_\Theta - \check{y}_\Theta) \\
L_{\Phi}(y_{\Phi} - \check{y}_{\Phi}) \\
\zeta[l_{1\Phi},l_{2{\Phi}}](y_\Phi - \check{y}_\Phi)
\end{bmatrix} +
\begin{bmatrix}
Bq_\Theta \\
0 \\
Bq_\Phi \\
0	
\end{bmatrix} +
\begin{bmatrix}
B(u_{r\Theta} + v_\Theta)\\
0 \\
B(u_{r\Phi} + v_\Phi) \\
0
\end{bmatrix},
\end{align}

and with corresponding output equations

\begin{align}
	\check{y}_\Theta &= \bar{C}_\Theta \check{x}_\Theta + \bar{D}_\Theta \Gamma_\Theta^* \\
	\check{y}_\Phi &= \bar{C}_\Phi \check{x}_\Phi + \bar{D}_\Phi \Gamma_\Phi^*		
\end{align}
	
Since matrices $C_i$ and $D_i$ differ from their nominal versions, the description for the derivatives of the states of the integral action for both inclination and azimuth are defined by (after substituting the output equations and considering the fact that $\check{x}_i = e_i + x_{ri} - \delta_i$):

\begin{align}
	q_\Theta' &= \zeta[l_{1\Theta},l_{2\Theta}](\Delta C_\Theta e_\Theta + \Delta C_\Theta x_{r\Theta} + \Delta D_\Theta \Delta \Gamma_\Theta^*  + \Delta D_\Theta \Gamma_{\Theta d}^* + \bar{C}_\Theta \delta_\Theta + EW_y), \\
	q_\Phi' &= \zeta[l_{1\Phi},l_{2\Phi}](\Delta C_\Phi e_\Phi + \Delta C_\Phi x_{r\Phi} + D_\Phi (\Delta \Gamma_{\Phi} + \Gamma_{\Phi d}^*)^*\frac{\sin \check{\Theta}}{\sin \Theta} - \bar{D}_\Phi (\Delta \Gamma_\Phi^* + \Gamma_{\Phi d}^*)  + \bar{C}_\Phi \delta_\Phi),
\end{align}

where $\Delta D_i = D_i-\bar{D}_i$ and $\Delta C_i = C_i - \bar{C}_i$. Afterwards, the observer error dynamics can be obtained. The main difference is that in this case, due to model uncertainty, the observer dynamics are not decoupled from the system dynamics (not even for the inclination). The complete system closed-loop error dynamics (for state vector $X(\xi)$ defined as in \eqref{eq:totaldynamics}) are given by

\begin{align}
	X'(\xi) =&	A_{0cl}X(\xi) + A_{1cl}X(\xi_1) + A_{2cl} X(\xi_2) + \nonumber\\ &P_{cl}(u_{r\Theta},u_{r\Phi},\Gamma_{\Theta d}^*,\Gamma_{\Phi d}^*,x_{r\Theta}(\xi),x_{r\Theta}(\xi_1), x_{r\Theta}(\xi_2),x_{r\Phi}(\xi),x_{r\Phi}(\xi_1),x_{r\Phi}(\xi_2), \Theta,\check{\Theta},W,W_y),
	\label{eq:totaldynamicsrobust}
\end{align}	

where:

\begin{equation}
A_{0cl} = 
\begin{bmatrix}
A_{0\Theta} & 0 \\
0 & A_{0\Phi}
\end{bmatrix}, \qquad
A_{1cl} =
\begin{bmatrix}
A_{1\Theta} & 0 \\
0 & A_{1\Phi}
\end{bmatrix}, \qquad
A_{2cl} =
\begin{bmatrix}
A_{2\Theta} & 0 \\
0 & A_{2\Phi}
\end{bmatrix},
\label{eq:ClosedLoopMatricesRobust}
\end{equation}

where the system matrices in \eqref{eq:ClosedLoopMatricesRobust} and the vector $P_{cl}(\xi,\xi_1,\xi_2, \Theta,\check{\Theta},W,W_y)$ (where all the dependencies on terms related to trajectory have been substituted by $\xi$, $\xi_1$ and $\xi_2$) are given by:
\begingroup
\renewcommand*{\arraystretch}{1.5}
\begin{align*}
A_{0\Theta} &= 
\begin{bmatrix}[cccc:cc]
A_0 & (B_{0\Theta} - B_{1\Theta} \bar{b}_0) & 0 & -B_{1\Theta}\bar{b} _1 & 0 & 0\\
%
0 & -\bar{b}_0 & 0 & -\bar{b}_1 & 0 & 0\\
\zeta \begin{bmatrix}k_{1\Theta} , 0 , 0\end{bmatrix} & 0 & 0 & 0 & -\zeta\begin{bmatrix}k_{1\Theta} , 0 , 0\end{bmatrix} & 0\\
%
\gamma K_\Theta & 0 & \gamma & -\gamma & -\gamma K_\Theta & 0\\ \hdashline
%
(\Delta A_0 - L_\Theta \Delta C_\Theta) & (B_{0\Theta} - B_{1\Theta} \bar{b}_0 - L_\Theta \Delta D_\Theta) & 0 & (-B -B_{1\Theta}\bar{b}_1) & \bar{A}_0 - L_\Theta \bar{C}_\Theta & -B\\ 
%
\zeta \begin{bmatrix}l_{1\Theta} , l_{2\Theta}	\end{bmatrix} \Delta C_\Theta & \zeta \begin{bmatrix}l_{1\Theta} , l_{2\Theta}	\end{bmatrix} \Delta D_\Theta & 0 & 0 & \zeta \begin{bmatrix}l_{1\Theta} , l_{2\Theta}	\end{bmatrix} \bar{C}_\Theta & 0
\end{bmatrix},\\
\nonumber \\
%
%
%
A_{0\Phi} &= 
\begin{bmatrix}[cccc:cc]
A_0 &(B_{0\Phi} - B_{1\Phi} \bar{b}_0) & 0 & -B_{1\Phi}\bar{b} _1\\
%
0 & -\bar{b}_0 & 0 & -\bar{b}_1 & 0 & 0 \\
%
\zeta \begin{bmatrix}k_{1\Phi} , 0 , 0\end{bmatrix} & 0 & 0 & 0 & -\zeta\begin{bmatrix}k_{1\Phi} , 0 , 0\end{bmatrix} & 0 \\
%
\gamma K_{\Phi} & 0 & \gamma & -\gamma & -\gamma K_{\Phi} & 0\\ \hdashline
%
(\Delta A_0 - L_\Phi \Delta C_\Phi) & (B_{0\Phi} - B_{1\Phi} \bar{b}_0 - L_\Phi (D_\Phi \frac{\sin \check{\Theta}}{\sin \Theta} - \bar{D}_\Phi)) & 0 & (-B -B_{1\Phi}\bar{b}_1) & \bar{A}_0 - L_\Phi \bar{C}_{\Phi} & -B \\
%
\zeta \begin{bmatrix}l_{1\Phi} , l_{2\Phi}	\end{bmatrix} \Delta C_\Phi & \zeta \begin{bmatrix}l_{1\Phi} , l_{2\Phi}\end{bmatrix} (D_\Phi \frac{\sin \check{\Theta}}{\sin \Theta} - \bar{D}_\Phi) & 0 & 0 & \zeta \begin{bmatrix}l_{1\Phi} , l_{2\Phi}\end{bmatrix}\bar{C}_\Phi & 0
\end{bmatrix},\nonumber \\
\nonumber \\
%
%
%
A_{1i} &= 
\begin{bmatrix}[cccc:cc]
A_1 & 0 & 0 & 0 & 0 & 0\\
%
0 & 0 & 0 & 0 & 0 & 0\\
0 & 0 & 0 & 0 & 0 & 0\\
%
0 & 0 & 0 & 0 & 0 & 0\\ \hdashline
%
\Delta A_1 & 0 & 0 & 0 & \bar{A}_1 & 0\\ 
%
0 & 0 & 0 & 0 & 0 & 0
\end{bmatrix}, \quad
%
%
%
A_{2i} = 
\begin{bmatrix}[cccc:cc]
A_2 & 0 & 0 & 0 & 0 & 0\\
%
0 & 0 & 0 & 0 & 0 & 0\\
0 & 0 & 0 & 0 & 0 & 0\\
%
0 & 0 & 0 & 0 & 0 & 0\\ \hdashline
%
\Delta A_2 & 0 & 0 & 0 & \bar{A}_2 & 0\\ 
%
0 & 0 & 0 & 0 & 0 & 0
\end{bmatrix}, \nonumber \\
\nonumber \\
P_{cl}&(\xi,\xi_1,\xi_2, \Theta,\check{\Theta},W,W_y) = \begin{bmatrix}
F_{p\Theta} + (-B - B_{1\Theta}\bar{b}_1)u_{r\Theta} \\
0 \\
0 \\
0 \\ \hdashline
F_{r\Theta} + (-B - B_{1\Theta}\bar{b}_1)u_{r\Theta}\\
\zeta \begin{bmatrix}l_{1\Theta} , l_{2\Theta}\end{bmatrix} (\Delta C_\Theta x_{r\Theta} + \Delta D_\Theta \Gamma_{\Theta d}^* + EW_y) \\ \hline
F_{p\Phi} + (-B - B_{1\Phi}\bar{b}_1)u_{r\Phi}\\
0 \\
0 \\
0 \\ \hdashline
F_{r\Phi} + (-B - B_{1\Phi}\bar{b}_1)u_{r\Phi}\\
\zeta \begin{bmatrix}l_{1\Phi} , l_{2\Phi}\end{bmatrix} (\Delta C_\Phi x_{r\Phi} + (D_\Phi\frac{\sin \check{\Theta}}{\sin \Theta} - \bar{D}_\Phi)\Gamma_{\Phi d}^*) \\		
\end{bmatrix}.
\\\nonumber
\end{align*}
\endgroup

The non-defined terms $F_{r\Theta}$ and $F_{r\Phi}$ are given by:

\begin{align}
F_{r\Theta} &= (B_{0\Theta} - B_{1\Theta} \bar{b}_0 - L_\Theta \Delta D_\Theta)\Gamma_{\Theta d}^* +  (\Delta A_0 - L_\Theta \Delta C_\Theta) x_{r\Theta} (\xi) + \Delta A_1 x_{r\Theta} (\xi_1) + \Delta A_2 x_{r\Theta} (\xi_2) + BW - L_\Theta EW_y \label{eq:PerturbationForces1obs},\\
F_{r\Phi} &= \bigg(B_{0\Phi} - B_{1\Phi} \bar{b}_0 - L_\Phi \bigg(D_\Phi\frac{\sin \check{\Theta}}{\sin \Theta} - \bar{D}_\Phi \bigg)\bigg)\Gamma_{\Phi d}^* + (\Delta A_0 - L_\Phi \Delta C_\Phi) x_{r\Phi} (\xi) + \Delta A_1 x_{r\Phi} (\xi_1) + \Delta A_2 x_{r\Phi} (\xi_2) \label{eq:PerturbationForces2obs},
\end{align}

where $\Delta A_0 = A_0 - \bar{A}_0$, $\Delta A_1 = A_1 - \bar{A}_1$ and $\Delta A_1 = A_1 - \bar{A}_1$.

\subsection{Linearization of robust system}









\end{document}