\documentclass[main.tex]{subfiles}

\begin{document}
	
\vspace{20pt}

In this section a linearization of the previously derived model is performed, this step is needed since the controllers that are going to be implemented are linear. Linearization is based on the Taylor Series, using only the first term to approximate the function. This means that the behavior of the nonlinear system will be similar to the linear one in a region close to the point where the function is evaluated. The point on which the system is linearized is based on the equilibrium points of the system, which are the regions where the derived controllers work. 

Considering only one full rotation of the system (since $\pi=2\pi=...=n\pi$ in the setup), The Rotational Pendulum has four equilibrium points, three unstable and one stable, which are:

\begin{eqnarray}
SE &=& \begin{bmatrix}
\pi & 0 & 0 & 0
\end{bmatrix} \nonumber \\
UE_1 &=& \begin{bmatrix}
0 & \pi & 0 & 0
\end{bmatrix} \nonumber \\
UE_2 &=& \begin{bmatrix}
0 & 0 & 0 & 0
\end{bmatrix} \nonumber \\
UE_3 &=& \begin{bmatrix}
\pi & \pi & 0 & 0
\end{bmatrix} \nonumber \\
\end{eqnarray}

$SE$ \nomenclature{$SE$}{Stable equilibrium.} stands for the stable equilibrium and $UE_i$ \nomenclature{$UE_i$}{Unstable equilibrium i.} stands for the unstable equilibria \textit{i}. It has to be taken into account, that due to the fact that in the setup the first link presents high friction, the equilibrium point $UE_1$, can also be consider as stable, despite the fact that in theory is unstable. In order to perform the linearization, a value to the input must be given. In this case the input $u=0$ will be used. The obtained equations of motion from section \ref{sec:Mathematical} will be linearized according to the following equation:

\begin{equation}
\ddot{q_l} = \ddot{q}(0) +
			 \frac{\partial \ddot{q}}{\partial q}\bigg\vert_{q = q_0}(q-q_0) +
             \frac{\partial \ddot{q}}{\partial u}\bigg\vert_{u = 0}(u)
\end{equation}

where $q_0$ \nomenclature{$q_0$}{Point at which the linearization will be performed.} is the point at which the linearization will be performed and $\ddot{q_l}$ \nomenclature{$\ddot{q_l}$}{Linearized equation for the angular acceleration.} is the linearized equation of the angular acceleration. Furthermore, after the linearization, the system is brought to state-space form using the following change of variables:

\begin{eqnarray}
x_1 &=& \theta_1 = q_1 \nonumber \\
x_2 &=& \theta_2 = q_2\nonumber \\
x_3 &=& \dot{x_1} = \dot{\theta_1} = \dot{q_1} \nonumber \\
x_4 &=& \dot{x_2} = \dot{\theta_2} = \dot{q_2} \nonumber \\
\end{eqnarray}

where $x_i$ \nomenclature{$x_i$}{State "i" of the system.} represents the state "i" of the system. As it can be seen, the system corresponds to fourth order. As before, the linearization was performed using the Symbolic Math Toolbox\textsuperscript{\textregistered}. An expression in terms of the parameters of the system was obtained, which can be found completely on appendix \ref{AppB:LinearizedModel}. The system has the following form:

\begin{equation}
\begin{bmatrix}
\dot{x_1} \\
\dot{x_2} \\
\dot{x_3} \\
\dot{x_4} \\
\end{bmatrix} = 
A 
\begin{bmatrix}
x_1 \\
x_2 \\
x_3 \\
x_4 \\
\end{bmatrix} +
B u
\end{equation} 

Since the system gives measurements for $\theta_1$ and $\theta_2$, these two will be taken as outputs of the system, and direct feedthrough is not considered in the system ($D=0$) the resulting equation is the following:

\begin{equation}
y = 
\begin{bmatrix}
1 & 0 & 0 & 0 \\
0 & 1 & 0 & 0 \\
\end{bmatrix}
\begin{bmatrix}
x_1 \\
x_2 \\
x_3 \\
x_4 \\
\end{bmatrix}
\end{equation} 


\end{document} 