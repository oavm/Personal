%
% Abstract (does not appear in the Table of Contents)
\begin{document}

\chapter*{Abstract}%

The amount of mineral and energy resources in conventional reservoirs has been drastically diminished in recent years. This has led to the development of techniques to reach reservoirs which are not easy to access and more specifically, where boreholes with complex geometries need to be drilled. Directional drilling allows to accomplish such complex boreholes. Although this process is not new to the industry, its complex dynamics have caused the development of different models generally described in a numerical way or by delay differential equations. Such models can be used to predict the evolution of a borehole or to develop control strategies. 

Current state-of-practice control methods are basically open-loop, where an operator has the task to steer the drilling system while trying to follow a reference for the borehole trajectory. This usually generates undesired behavior such as borehole spiraling and kinking. In order to provide a way to avoid these undesired effects, novel control strategies need to be designed to improve the directional drilling process. 

This literature study aims to provide an overview of the current models used to develop control algorithms, assess and compare such control strategies. Two types of models are presented: numerical/input-ouput models and first principles models. Regarding the latter, four main model descriptions are analyzed: kinematic models, the \acf{EFFSZM} model, the Neubert and Heisig model and the Perneder and Detournay (PD) model. Moreover, an overview of the controllers developed on the basis of these models is given and the document finalizes with an outlook on further research to be pursued with an MSc project by the author.

\end{document}


