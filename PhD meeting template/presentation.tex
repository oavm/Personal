% TU Delft beamer template
% Author: Erwin Walraven (initial version was created by Maarten Abbink)
% Delft Universiy of Technology

\documentclass{beamer}
\usepackage{amsmath}
\usepackage{amsfonts}
\usepackage{amsthm}
\usepackage{mathtools}
\usepackage[english]{babel}
\usepackage{calc}
\usepackage[absolute,overlay]{textpos}
\usepackage{graphicx}
\usepackage{subfig}
\usepackage{comment}
%\usepackage{tikz}
\usepackage{wasysym}
\usepackage{gensymb}
\usepackage{amssymb}
\usepackage{nccmath}
\usepackage{empheq}
\usepackage{xcolor}
\usepackage{relsize}
%\usepackage{apalike}
\usepackage[utf8]{inputenc}
\usepackage{multimedia}
\usepackage{media9}
%\usepackage{hyperref}.
\usepackage[table]{colortbl}


\usepackage{arydshln}


\makeatletter
\renewcommand*\env@matrix[1][*\c@MaxMatrixCols c]{%
	\hskip -\arraycolsep
	\let\@ifnextchar\new@ifnextchar
	\array{#1}}
\makeatother


\newcommand*\widefbox[1]{\fbox{\hspace{1em}#1\hspace{1em}}}
%\useoutertheme{miniframes}

%\usefonttheme[onlymath]{serif}



\setbeamertemplate{navigation symbols}{} % remove navigation symbols
\mode<presentation>{\usetheme{tud}}


%\ExecuteBibliographyOptions{parentracker=false}



% BIB SETTINGS
%\usepackage[backend=bibtex,firstinits=true,maxnames=30,maxcitenames=20,url=false,style=authoryear]{biblatex}
%\bibliography{MyBib}
%\addbibresource{MyBib.bib}
%\setlength\bibitemsep{0.3cm} % space between entries in the reference list
%\renewcommand{\bibfont}{\normalfont\scriptsize}
%\setbeamerfont{footnote}{size=\tiny}
%\makeatletter
%\renewcommand{\cite}[1]{\footnote<.->[frame][{\fullcite{#1}}]}
%\renewcommand{\cite}[1]{$[$\cite{#1}$]$}


% Insert frame before each subsection (requires 2 latex runs)
\AtBeginSection {
	\begin{frame}<beamer>[noframenumbering] \frametitle{\titleSubsec}
		\tableofcontents[currentsection,currentsubsection]  % Generation of the Table of Contents
	\end{frame}
}

% Define the title of each inserted pre-subsection frame
\newcommand*\titleSubsec{Outline}
% Define the title of the "Table of Contents" frame
\newcommand*\titleTOC{Outline}

\graphicspath{ {./images/} } 

\title[PhD meeting: Planning and Control]{PhD meeting \vspace{15pt}}
\institute[]{Istituto Italiano di Tecnologia, Genova, Italy \vspace{20pt}}
\author{Octavio A. Villarreal Maga\~na \vspace{20pt}} %\scriptsize Committee:  \\ Prof.Dr.Ir. Nathan van de Wouw (TUDelft, supervisor) \\ \vspace{1.5pt} \hspace{-11pt}Prof.Dr.Ir. Emmanuel Detournay (UMN, supervisor) \\ \vspace{-2.5pt} \hspace{-74pt}Dr.Ir. Manuel Mazo Jr. (TUDelft)}
\date{\today}

\begin{document}
{
\setbeamertemplate{footline}{\usebeamertemplate*{minimal footline}}
\frame{\titlepage}
\begin{frame}\frametitle{\titleTOC}
	\tableofcontents
\end{frame}
}

{\setbeamertemplate{footline}{\usebeamertemplate*{minimal footline}}
}


\section{Summary of work}
\begin{frame}{Summary of work}
\begin{itemize}\setlength\itemsep{3em}
\item Worked on implementation of max-plus in framework
\item Studied feedback linearization of Thiago Boaventura
\item Started convex optimization study group
\item Had a meeting to define goals towards paper
\end{itemize}
\end{frame}

\section{Work on max-plus gait generation}

\begin{frame}{Work on max-plus gait generation}
	\begin{itemize}\setlength\itemsep{3em}
		\item Current state: The gazebo simulation works the same as in the simulation in python that I previously had
		\item Further features: 
		\begin{itemize}
		\item Instantaneous (1 step) gait parameter change
		\item Problems with touchdown and lift-off event detection
		\item Angular frequency modulation
		\item Try a different trajectory generator
		\end{itemize}
	\end{itemize}
\end{frame}

\begin{frame}{"Instantaneous" gait change}
	\begin{itemize}\setlength\itemsep{1.5em}
		\item Detect legs in swing face
		\item Use them as the "follow-up" legs for the next gait pattern
		Example:
		If the current gait is:
		\begin{equation*}
		G_1 = \{1,4\}\prec \{2,3\}
		\end{equation*} 
		And we desire to change to:
		\begin{equation*}
		G_2 = \{1\} \prec \{2\} \prec \{3\} \prec \{4\} 
		\end{equation*}
		In the case that legs $2$ and $3$ are in the air, the gait transition patterns could be chosen as:
		\begin{align*}
		G_2^* &=  \{2\} \prec \{3\} \prec \{1\} \prec \{4\} \\
		G_2^{**} &= \{3\} \prec \{4\} \prec \{1\} \prec \{2\}
		\end{align*}
		\item The transition would be safe, would be made during the step that the robot is taking and with a finite number of gait transition patterns
	\end{itemize}
\end{frame}

\begin{frame}{Evaluation of viability of max-plus generation}
\begin{itemize}\setlength\itemsep{3em}
\item Run comparison against current method
\begin{itemize}
\item Gait switch
\item Versatility of gait generation
\item Difficulty to be implemented and controlled
\item Flexibility to be included in the framework
\end{itemize}
\item Analyze pros and cons between the two approaches
\item (Propose new approach)
\end{itemize}
\end{frame}


\section{Goals of the paper}

\begin{frame}{Goals of the paper}
\begin{itemize}\setlength\itemsep{3em}
\item Overcome terrain variations (such as a ramp) based on timed-event feedback 
\item Come up with a methodology to decide an implement in a systematic way, a set of gait parameters to overcome certain types of terrain (e.g., stair climbing)
\end{itemize}
\end{frame}

\section{A few announcements}

\begin{frame}{Announcements}
\begin{itemize}\setlength\itemsep{3em}

\item Machine Learning Crash Course will starts 26/06
\item BMVA Computer Vision Summer school starts 03/07 (have to pay for myself and reimbursed afterwards, since it can only be paid by credit or debit card)
\end{itemize}
\end{frame}

\begin{frame}
 Summer School on Foundations of Robotics and Autonomous Learning 
\begin{itemize}
\item Dates: 04/09 - 08/09 (close to ICRA deadline)
\item Location: Berlin
\item Topics: 
\begin{itemize}
\item Robotics: with a focus on foundations and a unifying perspective of the involved fields
    Basics of control, manipulation, planning

\item Machine Learning: with a focus on Autonomous Learning: the interaction of decision theory and learning
    Basics of Decision Theory, Active Learning, Bayesian Experimental Design, Reinforcement Learning, Adaptive Control, Inverse RL, Machine Learning applied to Robotics
\end{itemize}
\item Lecturers: Michael Beetz (Bremen University), Oliver Brock (TU Berlin), Sami Haddadin (Hannover University),Tamim Asfour (KIT), Ludovic Righetti (Max-Planck Institute for Intelligent Systems), Marc Toussaint (University of Stuttgart) 
\end{itemize}
\end{frame}

\begin{frame}
 \hspace{2cm} Thank you. Questions or comments?
\end{frame}





\end{document}
