\documentclass[../main.tex]{subfiles}

\begin{document}

There is no doubt, that there has been a growing interest in robotics in the last decades. Catastrophic events, such as the nuclear disaster in Fukushima have made evident,  that despite the technological advances achieved in recent years, both theory and practice are still far from developing robots that could face situations and environments which are hazardous to humans, with the same flexibility and adaptability that a human would display. Although robotic platforms can surpass humans and animals on certain tasks (fast computation and decision making, force demanding tasks such as lifting objects), there are some that biological beings perform with extraordinary easiness and elegance. Among these lies the problem of locomotion.

It has been observed that the performance of legged robots surpasses that of wheeled platforms while traversing rough or unstructured terrains. This feature comes at the cost of more difficult ways to model and control the dynamics of the agents involved in the process. This complexity has pushed research to develop advanced techniques in order to tackle the problem of robot locomotion. 

Several approaches have been taken to produce locomotion on legged robots. Some take advantage of the underlying mechanic properties of walking, others try to force stability by analyzing quasi-static solutions of the problem online, and the most advanced ones, have taken inspiration from biological systems to emulate walking patterns in robots. From the previously mentioned alternatives, the latter one, has proven to be the most robust against disturbances, and unstructured terrain, by effectively changing gait parameters depending on the situation that the robot is facing. Despite this, there are just a few examples where the agent is able to entirely switch the gait in order to perform a specific task.

All the previously mentioned strategies are based on continuous time models of locomotion. An alternative to this continuous models is the use of Discrete Event Systems (DES), an approach that has been taken in \cite{Lopes2010}, \cite{Barai}. In particular in \cite{Lopes2010}, the authors make use of a framework based on max-plus algebra systems. This strategy provides several advantages with respect to its continuous counterpart. One of the most notable is that the resulting description is a max-plus linear system, for which linear control techniques can be implemented in order to provide safe switching between different gaits. Due to this max-plus linear property, the computational effort is also reduced, allowing faster reaction times. At current time no paper related to this topic that involves gaits where the robot spends longer times with no support (for example gallop) has been published, although it is mentioned to be studied in the future.

This document proposes the implementation of a locomotion planning strategy based on Discrete Event Systems (DES) on the HyQ robotic platform. This is done by making use of a framework based on max-plus algebra, which offers the possibility to analyze and implement in a max-plus linear way, different gaits for several configurations of legged robots. The main purpose would be to adapt the max-plus algebra framework to generate time references for each leg, using them to design the trajectory of each leg. It is believed that this locomotion planning strategy can be implemented in the current reactive controller framework present in the HyQ robot \cite{Barasuola}, in substitution of or in combination with the Central Pattern Generator (CPG) already present in the system. Furthermore, other elements of the current framework could be improved, such as the torque control used for each of the joint angles in order to cope with uncertainty. One of the suggested strategies, is to investigate the influence on the performance if adaptive computed torque control is used. 

\iffalse
\begin{itemize}
	\item Importance of robot locomotion
	\item Advantages of legged robots over wheeled robots
	\item Nonlinear descriptions of CPG
	\item Current locomotion planning techniques do not perform with robustness, reliability and flexibility it is needed
	\item Gait switching problems
	\item importance of investigating new planning strategies to analyze the problem from a differente perspective
	\item why max-plus? Gait switching, variation of parameters more intuitive, less computationally expensive
	\item disadvantages, no work has been done on aerial phases
\end{itemize}
\fi
	
\end{document}
	