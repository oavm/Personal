\documentclass[main.tex]{subfiles}

\begin{document}
	In order to achieve the tasks proposed in the methodology a tentative work plan is designed. A brief summary of the plan is shown in Table \ref{tab:Planning}.
	
	\begin{table}[H]
		\caption{Tentative plan of work.}
		\centering
		\begin{tabular}{|l|l|}
			\hline
			Task &Time to fulfill task\\ \hline
			Literature research & 2-3 months\\ 
			Analysis of current implementation and platform  & 4-6 months\\ 
			Locomotion controller design & 7-9 months\\ 
			Simulation testing & 4-6 months\\ 
			Physical implementation & 5-7 months\\
			Unstructured scenario design and testing & 3-5 months\\
			\hline
		\end{tabular}
		\label{tab:Planning}
	\end{table}
	
	Initially, an extensive literature research has to be performed to strongly justify the methods and strategies proposed to achieved the desired objectives. All the suggested time frames include documentation of each of the steps taken, this meaning that the time to write down the results is being considered. Based on previous experiences, a literature research con take between 2 and 3 months, depending on how the topic has been studied. For this topic, the time frame seems reasonable, to both analyze and document the previous research. 
	
	Afterwards, an analysis of the previous work performed on the HyQ platform is in place. This analysis includes several subtasks. These can be divided in two main groups, theoretical and practical. Regarding the theoretical tasks, a crucial component is getting a deep understanding of the mechanical modeling of the robot. Furthermore, it is also needed to analyze deeply the current locomotion framework, to give solid reasons to proceed with the research. One should be able to reproduce all previous results. Also, during this part of the research, it has to be considered the need to learn the current tools being used to implement the locomotion controller design, i.e. unfamiliar programming languages and software. It is considered that this task can take between 4 and 6 months.
	
	The next step, is to proceed with the design of the new locomotion strategy based on the previously mentioned methodology. This is one of the most demanding and important tasks of the project. Analyzing the previous section, the max-plus algebra discrete time scheduler has already been implemented in several languages and softwares (python, MATLAB, V-REP) and could be easily adapted to other. The most demanding tasks would be adapting the time scheduler to the configuration of the HyQ robot. Based on the proposal in this last regard, it is considered that this task could take between 5 and 7 months. At this moment, taking into account the information gathered in the literature research and the locomotion strategy, it should be considered if there is time available to improve the motion controller, (i.e., implementing adaptive computed torque control in workspace), or if the current controller will be retained.
	
	Once the controller design has been achieved, the strategy has to be tested via simulation. Several tools can be used. Simulation can be done easily via Simulink and MATLAB, this being convenient since they will be useful as well for controller design. On the other hand, a step further would be to provide with 3D simulations of the robot executing tasks using the previously designed locomotion control. As mentioned in the methodology, V-REP and ROS could be use for this task. Also learning other tools such as Gazebo could be considered. This task could take between 4 and 6 months.
	
	Physical implementation is to be done after testing in simulation. It is considered that this task could be very demanding as well, since it also involves learning how to use the sensors and actuators of the platform. It is considered that this task can take from 5 to 7 months.
	
	Finally, as the end goal of this project is to design and implement a strategy which provides robustness to unstructured terrain as well as disturbances, it is important to carefully design the experiments that are going to be performed in order to fulfill this goal. it is considered that this task can take from 3 to 5 months.
	
	It is considered that the project could be completed in a time frame of 25 to 36 months. If there is time left, additional tasks could be considered.


\end{document}
