\documentclass[../main.tex]{subfiles}

\begin{document}

The analysis performed on the state of the art of the locomotion problem, has made clear that there are several open questions in the field. This has given important guidelines in order to propose the objectives that can be pursued in the research project on this topic. First of all, the robustness and flexibility of the current strategies has to be improved. Several alternatives have been studied in the previous section, but one that shows systematic implementation, is to use discrete event systems described by making use of max-plus algebra. One of the initial goals, is to prove if this framework can be implemented in a practical manner to robots with topologies different than the RHex inspired platforms. 

One subtask of this first objective, is to investigate if this locomotion strategy can be implemented in the reactive framework of the HyQ robot proposed in \cite{Barasuola}. This can be done either in tandem with the already implemented CPG trajectory generator, or by implementing a unique discrete max-plus gait scheduler. It is believed that the rest of the elements in this reactive framework, can be used in the same way as it is used making use only of a CPG. 

A second objective, would be to test if the use of max-plus algebra systems, indeed provides the alternative to perform gait changes, along with modifying the parameters of the gaits. A subtask of this objective may include the implementation of a supervisory controller, which uses the information from the sensors of the robot, in order to decide the most adequate gait and parameters to face each situation (in a similar way as in \cite{Barasuol}). 

Furthermore, it is believed that the currently implemented joint controller could be improved if an adaptive scheme is added, which could provide online estimations of the model parameters. This could be helpful to improve robustness, since in most dynamical system the parameters that describe the model could change due to several factors (unmodelled or neglected dynamics, disturbances).

This objectives have to be tested on both simulation and the real system. 

\iffalse
\begin{itemize}
	\item Describe objectives with respect to robust locomotion generation (sub-objectives)
	\item Describe objectives with respect to implementation of max-plus in current system (sub-objectives)
\end{itemize}
\fi 

\end{document}