\documentclass[../main.tex]{subfiles}

\begin{document}
	
	In robotics, the problem of legged locomotion has proven to be a challenging task. There are three particular areas where most of the issues are encountered: control, mechanics and electronics. Regarding control, the dynamics of a walking robot can be seen as a hybrid system, due to the fact that it involves continuous control of each of the walking limbs, but in coordination with the event triggered movements of the other limbs. More importantly, in most cases, the action of balancing (which humans and animals perform naturally) represents an intrinsically unstable equilibrium point \cite{Holmes2006}. Considering the mechanics, apart from the underlying properties of the different robotic platforms, modeling and providing proper alternatives to cope with the interactions of the platform with the environment (such as impact and slipping) is still difficult. Finally, performing gaits is a demanding task by itself, which demands high amounts of power, which could limit the autonomy of the robot. This problem is intimately related to the evolution of the mechanics in robotic systems, since the problem has been tackled from the point of view where passive-walking can be achieved without the use of any external source of energy \cite{Wisse2004}, or the development of actuators that provide enough force and torque while optimizing power consumption \cite{Tsagarakis2013}.
	
	Several approaches to achieve locomotion have been researched. One of the first steps towards accomplishing successful walking, was by using the concept of Zero-Moment Point (ZMP), which is defined as "\textit{the point on the ground at which the net moment of the inertial forces and the gravity forces has no component along the horizontal axes}" \cite{Borovac2004}. Using this concept, traditionally one could design a controller that successfully moves the end effector while maintaining the ZMP inside the support foot, by means of tools like inverse kinematics (which yields several disadvantages such as singularity points and multiple solutions) or by implementing an operational space formulation with constraints \cite{Aghili2005}. This framework has been applied to some robotic platforms such as Honda's Asimo. Nevertheless, in reality it yields several problems, as it does not represent a natural way of walking, and due to the fact that the resulting equations for the dynamics of the robot can be very complex, it may require a lot of computational power and may not be robust to large perturbations. Furthermore, it considers a quasi-static movement, so running yields a high level of complexity as well. Finally, as the number of legs increases, the problem becomes less tractable.
	
	A different way of tackling the problem is to model the body of the human or animal as a point mass located at the hip joint with two massless legs, equipped with telescopic springs with constant stiffness and length. This model is called the Spring Loaded Inverted Pendulum (SLIP) and it has been shown that it represents both walking and running reliably \cite{Geyer2006} and it can reproduce the ground reaction forces and hip trajectories observed in gaits. Moreover, it can encode various gaits by defining different model parameters (length of the leg, stiffness of the spring, angle of attack of the foot), and can execute energy efficient passive gaits (in the case of bipeds).  A step has been taken further by proposing a Variable Spring Loaded Inverted Pendulum (V-SLIP) model \cite{Ketelaar}. This is similar to the way humans control the actuation of their limbs via the muscular system, changing their stiffness in order to achieve movement. This strategy makes use of energy-based Hamiltonian mechanics to design a controller to achieve stable walking using different gaits. These two models have been proven to be very suitable to describe walking and running in humans and animals. Despite this, they face the same problem of complexity when it comes to controller design.
	
	One of the most widely utilized tools in robot locomotion, are the so-called CPGs. The problem of animal and human locomotion had been widely studied long before robotics were a relevant subject. The interest on this topic initially arose from the field of biology, achieving great success in understanding how animals such as mammals and insects manage to coordinate in a natural and effortless way the movement of their legs in order to perform specific gaits corresponding to the conditions they face on a daily basis. It was discovered, that complex networks of neurons are in charge of generating the signals to send to each walking limb, using advantageously the periodic nature of the process. CPGs do not actuate each limb independently or are entirely based on the senses of animals/humans. This is because in most of the scenarios, walking constitutes a periodic activity, where the legs move following the same patterns with phase differences. Depending on the situation that the agent is facing, it naturally changes the parameters of the gait (such as the time that the legs remain on the ground or the air or the phase differences between each leg) or even changes the gait completely (such as going from walking to running).
	
	Roboticists around the world, have payed close attention to these neuronal arranges in order to apply them in robot locomotion. This type of structures have been emulated by using simplified models of neurons and their interactions. Using this kind of framework, various types of gaits can be achieved. The way that CPGs are used, differs among all different platforms, and has proven to be efficient to generate different gait motions for several types of robots \cite{Crespi1985}, \cite{Ijspeert2008}, \cite{Barasuola}. Some examples \cite{Ijspeert}, do not use any kind of feedback control in order to generate gaits, although the main purpose of this kind of study is to analyze the animal behavior using robotic models to simulate gait patterns. On the other hand, CPGs can be used to effectively generate angle references for each of the legs in robotic platforms \cite{Barasuol}, to subsequently be tracked via control strategies (the most widely used in humanoid and animal-like platforms are computed-torque control and feedback linearization techniques). 
	
	Despite the elegance that locomotion generation using CPGs displays, roboticists are facing big challenges regarding robustness against unstructured terrain and dealing with disturbances. In most cases, when encountering unfavorable situations, proprioceptive (information coming from the agent itself) and exteroceptive (information coming from the environment) sensory information, is used to change the parameters of the gait such as coupling variables, ground and flight times and velocities of each leg. This framework has proven to efficiently deal with complex types of terrain. In some specific cases, gait switching has been achieved \cite{Ijspeert}, although no control is implemented in the locomotion planning routines. It can be seen that even in the simplest case, a set of nonlinear differential oscillator equations describes the evolution in time of each limb. This is due to the intrinsic periodicity of the process. This provides an extra difficulty in order to change from one gait to another, especially as the number of legs increases. 
	
	All previously described strategies, model the locomotion problem as a continuous-time system. Another way to approach the locomotion planning problem, is to model it via Discrete Event Systems \cite{Schutter2009}. In recent years the use of tools such as petri nets and max-plus algebra systems have been tested and proved to be able to generate locomotion patterns. Max-plus algebra systems have provided not only a simpler and systematic way to generate gait motions but also, due to its linear (in max-plus) properties, provides tools to analyze transitions and velocity of the generated planning.
	
	Max-plus algebra is a type of "tropical algebra" that complies with the axioms of an algebraic semi-ring \cite{Schutter2008}. Max-plus algebra represents a very useful tool to model timed events such as railway scheduling \cite{Goverde2007} or traffic control \cite{Turek2011}. Robot locomotion can be modeled as well as a discrete event system, making use of a subclass of petri nets called timed event graphs \cite{Schutter2009}. Subsequently, one can represent the evolution equations for a subset of timed event graphs as max-plus algebra linear systems. 
	
	Using this kind of framework, one could generate time references for each of the leg's touchdown times (time instant when the leg touches the ground) and lift-off times (time instant when the leg leaves the ground). Subsequently, with these time references for each leg, a trajectory can be computed and achieved by means of a controller. It has been proven that this strategy can safely switch between gaits, minimizing the variation in velocity of all the legs touching the ground. It yields the disadvantage that it has been implemented in robots, with rather simple dynamics (RHex-inspired platform), although in theory, complexity in the mechanics of the platform would only represent a difference in the generation of references to be followed by the controller for each of the legs.
	
	In summary, robust and flexible locomotion planning remains an open question in the robotic community. Finding the right balance between complexity and tractability of the studied strategies represents one of the most important trade-offs in the field. The introduction of the concept of ZMP, gave a first glance at the complexity of the locomotion problem, making evident that novel strategies should be developed in order to render the problem tractable. On one hand, accurate models of both human and animal locomotion have been found throughout the past decades, which capture the essential dynamics of the process (such as the SLIP/V-SLIP models). These models have been widely utilized for biped locomotion, achieving great success. But, as the number of legs increases, the underlying non-linearities and the combinatorial space for coordination, render the task more difficult. The use of CPGs has aid regarding coordination of the legs, elegantly modeling the phase differences of each limb in order to achieve different types of gaits. Nevertheless, it faces the same problem regarding non-linearity and combinatorial "gait space". Finally, modeling the problem as a discrete event system using max-plus algebra presents itself as a simple and systematic way to coordinate the legs in order to achieve gaits and, furthermore, can provide safe gait switching. It remains to investigate if this framework can be applied to mammal-like walking limbs, made of several joints. 
	
		
	
\iffalse	
\begin{itemize}
	\item Explain HyQ framework broadly
	\item Explain CPG
	\item Introduction to max-plus
	\item Max-plus in locomotion
	\item Advantages and disadvantages of max-plus
	\item Discussion
\end{itemize}
	
	
\begin{itemize}
	\item How is locomotion done at this time?
	\item Continuous models (Central pattern generators)
	\item Discrete time-event system (Max-plus algebra systems)
	\item Introduction max-plus
	\item Max-plus in locomotion
	\item HyQ decription
\end{itemize}

\fi


\end{document}

