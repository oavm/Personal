\documentclass[a4paper]{article}

\usepackage[english]{babel}
\usepackage[utf8]{inputenc}
\usepackage{amsmath}
\usepackage{graphicx}
\usepackage[colorinlistoftodos]{todonotes}
\usepackage{float}
\usepackage{listings}
\usepackage{color}
\usepackage{fullpage}
\usepackage{amsmath, amssymb, graphics, setspace}
\usepackage{subcaption}
\usepackage{subfiles}
\usepackage{mathtools}
%\usepackage[framed,numbered,autolinebreaks,useliterate]{mcode}
\usepackage{nomencl} 
\usepackage[export]{adjustbox}
\usepackage{etoolbox}
\usepackage[nottoc,notlot,notlof]{tocbibind}
\usepackage{url}
\usepackage{multicol}
\usepackage{subcaption}
\usepackage{dcolumn}
\usepackage{amsmath}
\usepackage{amsfonts}
\usepackage{amssymb}
\usepackage{arydshln}
\usepackage{graphicx}
\usepackage{amsmath, amssymb, graphics, setspace}
\usepackage{subcaption}
\usepackage{enumerate}
\usepackage[utf8]{inputenc}
\usepackage[T1]{fontenc}
\usepackage{textcomp}
\usepackage{gensymb}
\usepackage{amssymb,amsmath}
\usepackage{tikz}
\usepackage{relsize}



\newcommand{\circleland}{ 
	\mathbin{
		\mathchoice
		{\buildcircleland{\displaystyle}}
		{\buildcircleland{\textstyle}}
		{\buildcircleland{\scriptstyle}}
		{\buildcircleland{\scriptscriptstyle}}
	} 
}

\newcommand\buildcircleland[1]{%
	\begin{tikzpicture}[baseline=(X.base), inner sep=0, outer sep=0]
	\node[draw,circle] (X)  {$#1\land$};
	\end{tikzpicture}%
}

%\renewcommand{\familydefault}{\sfdefault}
%\usepackage{helvet,sfmath}

%\usepackage{lmodern}
%\usepackage{bm}
%\usepackage{sansmath}
%\usepackage{cmbright}

\def\changemargin#1#2{\list{}{\rightmargin#2\leftmargin#1}\item[]}
\let\endchangemargin=\endlist

\makeatletter
\renewcommand*\env@matrix[1][*\c@MaxMatrixCols c]{%
	\hskip -\arraycolsep
	\let\@ifnextchar\new@ifnextchar
	\array{#1}}
\makeatother

\title{Leg locomotion control using max-plus algebra systems}

\author{O.A. Villarreal Maga\~na}

\date{\today}

\graphicspath{ {./img/} } 

\begin{document}

\maketitle
\thispagestyle{empty}
%\clearpage
\setcounter{page}{1}
% To prevent a page number on the bottom of the page
%\newpage
%\tableofcontents % To automatically create a table of contents from the \section and \subsection commands.
%\newpage


\thispagestyle{empty} 
\thispagestyle{empty}
%\newpage




% How to use the nomenclature 
 
 
%\newpage
%\section*{Overview}
%\subfile{sections/overview.tex}

%\section*{Planning}
%\subfile{sections/planning.tex}

%\section*{Introduction}
\subfile{sections/Introduction.tex}

%\newpage
\section{Motivation and rationale}
\subfile{sections/Motivation.tex} 

%\newpage
\section{State of the art}
\subfile{sections/Stateoftheart.tex} 

%\newpage
\section{Objectives}
\subfile{sections/Objectives.tex} 

%\newpage
\section{Methodology}
\subfile{sections/Methodology.tex} 

%\newpage
\section{Tentative workplan}
\subfile{sections/Workplan.tex} 

%\newpage
\section{Expected results}
\subfile{sections/Results.tex} 

%\newpage
%\section{Questions}
%\subfile{sections/questions.tex} 



%========================== Back matter ======================================

%
% Bibliography
\newpage
\bibliographystyle{ieeetr}
%\printbib{MyBib}
\bibliography{MyBib2}{}
%\bibliographystyle{plain}
%
%%
% Glossary
\chapter{Glossary} %
%
\printacronyms
\begin{acronym}[\hspace{0.8in}] % 0.8in is also used by the nomenclature
	\acro{3mE}[3\textlarger{m}E]{Mechanical, Maritime and Materials Engineering}%
	\acro{AMS}{American Mathematical Society}%
	\acro{DCSC}{Delft Center for Systems and Control}%
	\acro{TU}[TU D\textlarger{elft}]{Delft University of Technology}%
	\acro{RSS}{Rotary Steerable System}
	\acro{ROP}{Rate of Penetration}
	\acro{BHA}{Bottom Hole Assembly}
	\acro{MWD}{Measurement While Drilling}
	\acro{PDC}{Polycrystalline Diamond Compact}
	\acro{EFFSZM}{Explicit Force, Finitely Sharp, Zero Mass}
	\acro{ISISZM}{Infinitely Stiff, Infinitely Sharp, Zero Mass}
	\acro{FSFSZM}{Finitely Sharp, Finitely Stiff, Zero Mass}
	\acro{MPC}{Model Predictive Control}
	\acro{SISO}{Single Input Single Output}
	\acro{MIMO}{Multiple Input Multiple Output}
	\acro{RGA}{Relative Gain Array}
\end{acronym}%
%
%
% Nomenclature
\printnomenclature[3.5cm]%

%
% Index
\cleardoublepage


\end{document}