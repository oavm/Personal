\documentclass[main.tex]{subfiles}

\begin{document}
	
\vspace{20pt}
\subsection{Planning}

In order to set up a plan considering the amount of time that it will take to complete the thesis project, initially a tentative structure for the final document is proposed. 

\begin{enumerate}
	\item Abstract
	\item Introduction
	\item 3D Directional drilling model
	\item Open-loop dynamics
	\item Controller Design
	\item Conclusions and recommendations
\end{enumerate}
   
With this structure in mind, the following table for the planning is proposed:
   
\begin{table}[H]
\caption{Planning.}
\centering
\begin{tabular}{|l|l|l|}
\hline
Task &Time to fulfill task& Date of completion  \\ \hline
Closed-loop error dynamics of neutral bit walk model & 5 days &  21/03/2016\\ 
Controller synthesis neutral bit walk  & 3 days & 24/03/2016\\ 
Stability analysis and simulations of neutral bit walk model&4 days & 28/03/2016\\ 
Controller simulations neutral bit walk& 2 days & 30/03/2016\\ 
Write results of controller & 5 days & 04/04/2016\\
Robustness analysis of neutral bit walk model controller &1 week & 11/04/2016\\
Write  results of robustness analysis & 5 days & 16/04/2016\\
Closed-loop error dynamics of non-neutral bit walk model &2 weeks & 30/04/2016\\ 
Controller synthesis of non-neutral bit walk model &3 days & 03/05/2016 \\ 
Stability analysis and simulations of non-neutral bit walk model &4 days & 07/05/2016 \\ 
Controller simulations of non-neutral bit walk model &2 days & 09/05/2016 \\ 
Writing results of non-neutral bit walk controller &5 days & 14/05/2016 \\
Robustness analysis of non-neutral bit walk controller & 1 week & 21/05/2016\\
Writing results of robustness analysis & 5 days & 26/05/2016\\
Write model and open loop dynamics section & 3 days & 29/05/2016 \\
Writing conclusions, recommendations, abstract and introduction & 5 days & 03/06/2016\\ \hline
\end{tabular}
\label{tab:Planning}
\end{table}

A tentative date to have a draft of the report is the 3rd of June.

\begin{table}[H]
	\caption{Daily planning.}
	\centering
	\begin{tabular}{|l|l||}
		\hline
		Task & Date\\ \hline
		Develop Robust error dynamics & 04/09\\ 
		Analyze matrices  & 04/10 \\ 
		Synthesize controller and simulations & 04/11\\ 
		Write down results of robustness and correct previous report & 04/12 - 04/15\\ \hline
	\end{tabular}
	\label{tab:Planning}
\end{table}



\newpage
\section{Colloquia, workshops and presentations}

In this section the complete list of colloquia and workshops attended to will be given, a total of 10 colloquia out of 15 is considered, as well as 5 workshops out of 7. It has also to be taken into account that I have already given my literature colloquium, remaining the final thesis presentation to be given and 2 workshop presentations.

\subsection{Colloquia}

\begin{enumerate}
	\item Date: October 9th, 2015. \\
		Location: TUDelft.\\
		Topic: Model predictive control for efficiency improvement of a gas to liquids pilot plant.\\
		Presenter: Leon Kuiper.
		
	\item Date: November 11th, 2015.\\
	Location: TUDelft.\\
	Topic: GPU implementation of D-SABRE, a spline based wavefront reconstruction method for large-scale AO systems.\\
	Presenter: Niels Tielen.
	
	\item Date: January 22nd, 2016.\\
	Location: TUDelft.\\
	Topic: Robust urban traffic control.\\
	Presenter: Dik Jansen.
	
	\item Date: January 22nd, 2016.\\
	Location: TUDelft.\\
	Topic: Hybrid human in the loop model predictive control for urban drainage systems.\\
	Presenter: Franka Veltman.
	
	\item Date: January, 27th 2016.\\
	Location: TUDelft.\\
	Topic: Stochastic Distributed Coordination of Energy Balance Smart Thermal Grids.\\
	Presenter: Wayan Wicak Ananduta.
	
	\item Date: March 11th, 2016.\\
	Location: TUDelft.\\
	Topic: Propulsion control assesment by a system modelling approach\\
	Presenter: Max van de Leijgraaf.
	
	\item Date: March 11th, 2016.\\
	Location: TUDelft.\\
	Topic: Dynamic wind farm control for a sparse quasi LPV descriptor model.\\
	Presenter: Joeri Frederik.
	
	\item Date: October 14th, 2015.\\
	Location: TU Eindhoven.\\
	Topic: Model-based decoupled control of a 3D directional drilling system.\\
	Presenter: Frank Monsieurs.
	
	\item Date: December 4th, 2015.\\
	Location: University of Minnesota.\\
	Topic: A nonlinear dynamical model of borehole spiraling.\\
	Presenter: Julien Marck.
	
	\item Date: December 8th, 2015.\\
	Location:  University of Minnesota\\
	Topic: Nonlinear model of a 2D directional drilling system.\\
	Presenter: Bart Heijke.
	
\end{enumerate}

\subsection{Workshops}

\begin{enumerate}
	\item Date: September 30th, 2015.\\
	Location: TUDelft.\\
	Topics and presenters: 
	\subitem Oscar Sondermeijer. Regression-based inverter control for power flow and voltage regulation.
	\subitem Frank Warffemius. Iterative learning control:feasibility and implementation for a reticle stage.
	\subitem Matteo Ciocca. Simple models for robot interaction.
	\subitem Pascal Zijlstra. Control-based cryptography for a networked control system.
	\subitem Filip Paszkiewicz. How to model a generic short stroke actuator?
	\subitem Lex Blenkers. Railway disruption management.
	
	\item Date: October 30th, 2015.\\
	Location: TUDelft.\\
	Topics and presenters: 
	\subitem Filip Paszkiewicz. Control of generic short stroke actuator.
	\subitem Raziel Gonzalez. Collision avoidance with haptic feedback for telepresence.
	\subitem Pascal Zijlstra. Control based cryptography for a networked control system.
	\subitem Nelleke van der Steen. Ship motion predcition for the ampelmann system.
	\subitem Lex Blenkers. Railway disruption management.
	\subitem Alfons Schure. End effector dynamics control of parallel manipulators for haptic feedback.
	
	\item Date: January 29th, 2016.\\
	Location: TUDelft.\\
	Topics and presenters: 
	\subitem Jelle Munk. Learning a state representation.
	\subitem Ruben Burger. Dynamic obstacle avoidance of a 6DoF robotic manipulator.
	\subitem Bender Botterman.
	\subitem Guangshuo Xin. Parameter estimation of a human-like cybernetic model.
	\subitem Frank de Winkel. Engineering the distributed robotcis lab v2.0.
	\subitem Matteo Ciocca. Simple models for robot interaction.
	
	\item Date: February 24th, 2016.\\
	Location: TUDelft.\\
	Topics and presenters:
	\subitem Jessica Bautista. Use of the nuclear norm for system identification.
	\subitem Frank de Winkel. Distributed Formation Control using consensus and spatial predictive control.
	\subitem Wour Feitsma. Design of an above water position determination system for a ship's hull maintenance robot.
	\subitem Jelle Munk. Computing in the cloud.
	\subitem Danny Hammeeteman. Convey information through non-verbal communication in robotics.
	\subitem Niels Tielen. Implementation of a distributed spline based wavefront reconstruction method for extremely large telescopes on graphics processing units.
	
	\item Date: December 11th, 2015.\\
	Location: University of Minnesota.\\
	Topics and presenters: 
	\subitem Anna Liakou. Linear complementarity problem.

\end{enumerate}




\end{document}